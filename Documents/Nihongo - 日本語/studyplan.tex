Read The Kanji -- don't use this for kanji.
Make a free account, use it to learn the Hiragana and Katakana (two of Japanese's three alphabet systems; 48 characters each and phonetic. One is for Japanese-origin words, the other is for loan words and other random things). It just throws flash cards at you with each of the symbols; you can probably commit them to memory in a few hours. It's okay if you forget a few or several or even most of them at first; you're going to see these things so often that they'll be impossible to forget before long. We're just shooting to prime your passive memory so that you'll see a word written, have your curiosity irked, and be able to work it out, connecting that forgotten information to more and more recent memories to help remember them. Plus, this is a model for your year as a whole -- contextually acquiring passive understanding that stretches your boundaries, then diving back inwards and working to solidify passive knowledge that has become useful for your current situation or will allow you to express something you want to express currently, into knowledge that gradually becomes active.

Buy Genki I, its workbook, Genki II, and its workbook.
This will walk you from knowing absolutely no Japanese at the beginning of Genki I, and while mileage varies, I was personally able to make sense of ShiroKuma Cafe (see the link in the next section) upon completing Genki II. I'm currently taking the first "advanced" level Japanese course at my uni, meaning that I have had the opportunity to talk with other "advanced" (apostraphes meaning take with a grain of salt, looking at myself) learners about how they learned Japanese, and the Genki series is by and large the crowd favorite.

Buy Heisig, or you can probably find a version somewhere on the interwebs. 
Make an account at Kanji Koohii (a site where people work together progressing through Heisig, mainly by sharing the mneumonics they make for the kanji), and otherwise follow the instructions on Nihongo Shark's Blog. He suggests to completely put learning Japanese on hold till you finish the 2200 Kanji in this deck in 97 days, but I think that's ambitious as is, and eats too much of your year up. So I personally would say learn 15 a day, every day, until you finish -- that will have you finishing in around 5 months, you'll be on target with the 6 months I'm plotting out for Genki I + II even if you miss a few days. (see below).

Others might disagree and you can make up your own mind, but I personally think learning the Kanji is essential. They take time to learn at first, but repay you dividends later on when you accumulate vocabulary basically without thinking, passively, by reading or watching subtitled shows. Plus, any resource you'll use past the beginner stage will require kanji.. meaning if you don't learn them, you can't use these resources, and gimp yourself down the road. They're incredibly logical and like legos; the resources in #3 basically talk about the most efficient way to build things out of those legos (to help remember what each lego is). Also look into Moonwalks with Einstein if you'reinterested in memory in general. The thing about Kanji is that they unlock Japanese, as every single Kanji has a unique meaning, and Japanese words are basically simple definitions of themselves. Take fire extinguisher, for example: 消火器。It literally means extinguish-fire-utensil/tool. Good luck understanding a random word like that in any other language at first sight, but it's easy in Japanese, and the vast majority of Japanese words are exactly like this. Learning the Kanji allows you to take a word you've never seen before, instantly have a reliable guess as to what it means... and depending on your familiarity with the Kanji, maybe even how to read it. This happens to a lesser extent in conversation, also. Kanji are a new system of logic, but once you adjust to it, it's pure magic -- eventually, you sort of stop needing to study vocabulary, because you can just read and passive understand most any word (which you'll eventually work into your active vocabulary). I talk about "The First 2000 Words" in #5, and basically, words give you diminishing returns -- they're a lot of bang for your buck at first.. but past 6,000, 10,000, 20,000 ... learning 10 or 100 or even 1,000 new words might not give you noticeable improvement.


This anki deck is Genki in Example Sentences; pace your daily reviews so that you'll be going in time with your progression through chapters in the book. I really, really wanted to link you The Core 2k(the first 2000 most frequent words of Japanese) because I really liked it and the first 2000 words make up a significant majority of daily conversations (we repeat a lot of the same things over and over, the same bread and butter structures, laced and spiced with more rare nouns, then descriptive words, and the occasional verb)......... but I also think that context is the biggest key when it comes to language learning, and the 2k doesn't have that for you right now. It's eventually going to outpace your Kanji studies (if I'm recalling how I studied accurately), and more importantly, the word order does not follow Genki. You're going to be spending a lot of time with Genki for 6 months, the pace that I want you to complete these words in. You're already going to be stretched thin, so I guess I'm going to recommend you take that Genki deck and use it as a supplement to help you get more out of Genki -- it looks like it's going to take, on average, ~25 cards per day. I don't know if that's ideal, but then again, I stuck with Genki until I finished Genki (no other resources, began Hesig - also below - about 2/3 of the way through), and I began watching Shirokuma Cafe (below) immediately after Genki II, able to (at first, painfully) understand it... and I think I'm just a normal dude, if you're also a normal dude -- or, better, a better than average dude -- I guess Shirokuma should be good for you, too, after Genki II and this Genki Deck.

Post Genki II Stuff

Watch Shirokuma Cafe on this website.
Animelon is beautiful because all of its anime have subtitles available in English, romaji (latinized Japanese), hiragana, and normal Japanese -- start with English & normal Japanese for a few episodes to get used to how people talk, then turn off English and begin ganbatte'ing (doing your best). This anime is about a panda bear working in a cafe owned by a polar bear where they make food for guests and go on various adventures. It's great because the vocabulary is almost entirely every day (minus the polar bear's obnoxious puns), and it also has a variety of accents, so you'll begin getting used to Japanese sounds. If you like dry humor, you'll even enjoy the anime. I personally laughed so hard that I cried, twice.

Begin going through the N3 grammar videos from Nihongo no Mori, also feel free to check out their Dangerous Japanese (slang), and move on to N2 and N1 grammar as you feel ready. Their videos are great because they all have subtitles, they circumlocate to simpler Japanese to explain difficult words in the example sentenecs (explaining Japanese with simpler Japanese), and they have fun. These videos were personally the first "all Japanese" content that I consumed, and after I had been watching for a week or so I began with Shirokuma Cafe.

Buy Read Real Japanese Contemporary Fiction and Essays.
These books are great: they present 7 short stories or essays that are 100 unaltered (except for adding readings to Kanji that appear for the first time in a given article), as a native speaker would see them. That's on the right page. The left page has a running gloss into English -- it's just enough to help you understand the meanings of parts you didn't quite understand, but not so much that you'd understand what was going on by only reading it. The real gem is that the 2nd half of the book is a running grammatical dictionary, as in the author devotes like ~130 pages to explaining all of the grammar that was contained in every single article that is more advanced than ~Late Genki II stuff. These are the holy grail of Japanese learning content for me; they're literally training wheels for reading read Japanese stuff. I read each one with a notebook: I went one sentence at a time, reading every grammar explanation, and writing down any grammar that I didn't know. Sounds time consuming, but I still went through a story in 1-2 days (2-4 hours? per story on average). After finishing the book I waited 2 weeks then read it again, highlighting the sentences that I still struggled with, double checking that grammar. Then I read it again a month later, not checking the grammar, and added any sentences i still didn't explain into Anki as Clozed Deletion Card.

Read Real Japanese is training wheels to Reading Real Japanese.
Written Japanese is quite different than Spoken Japanese, and this book really helps to iron out everything that might have not quite gotten through your system yet. When you finish the two books, begin looking for native books you can read on an e-reader/the computer. Just pick whatever you're interested in that has been written in the last 20 years. It's important to do it on a Kindle/computer because this enables you to highlight words to search them in the dictionary, rather than having to draw the characters out to search by hand in your phone dictionary. The Kindle is a pair of stilts that makes reading tolerable at a fluency level where it would normally be unbearable -- and I think this goes for any language, but particularly for languages like Japanese/Chinese where the primary writing system isn't necessarily phonetic.

In addition to reading, listen to lots of stuff. Find something that is interesting to you -- ie, something you find entertaining enough that you're willing to slodge through the beginning phase where it's not-pleasantly-difficult -- and stick to it. I personally liked/like Taigu Channel; a Buddhist monk here in Japan takes in letters from people struggling with life problems (what is happiness? what is freedom? How can I show the people around me that I appreciate them?) and then he answers them from a Buddhist perspective. Objectively speaking I think it's super for a first listening resource because he speaks clearly, somewhat slowly, a lot of the videos have subtitles, and he's talking about everyday-life problems meaning that the vocabulary is limited to practical things. If you're interested in Buddhism, I personally find the videos to be really enlightening. This is the ultimate goal of language learning, in my opinion -- to find a way to make your target language a medium; a gateway to knowledge or entertainment that you want, which just happens to be only in your target language... meaning that just by enjoying yourself and consuming content you want to consume, you naturally improve your language.

